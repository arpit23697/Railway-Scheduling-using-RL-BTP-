\chapter{Simulator}

This module is responsible for carrying out the whole simulation and
 putting all the components in place. The way it does this is by creating
 processes that interact with each other and runs the simulation.
 This module is implemented with the help 
 of simpy that helps to create different processes. SimPy is a discrete-event simulation 
 library. The behavior of active components (like trains, deadlock detection or creating graphs) 
 is modeled with processes. All processes live in an environment. They interact with the environment 
 and with each other via events (which is created by this module). Note all the processes are 
 running \textbf{concurrently}. At last, Simpy is using priority queue to 
 order the events. There is a clock in the environment and it is the simulator that runs the clock, 
 esentially running the simulation. 

\vspace{0.25cm}

Simulator module first create the network (with the help of railway network component) and then
the \textbf{environment} under which the simulation is carried out. Then it creates 
various processes that runs in this environment. The processes are : 
\begin{enumerate}
\item \textbf{Trains}

There are multiple trains which are running in the network. Each train is an instance of 
train class implemented in the train component. Simulator creates each train as a process. These
trains are running over the same resource pool (railway network) and simulator helps in scheduling and 
running each train. Each train have two actions, either to move or to wait and these 
actions are implemented using \textbf{choose action} process.

\item \textbf{Choose action}

This process always runs and take actions for each train in the network.
Initially there are no trains in the network. Simulator puts them at the initial station at approriate time 
(depending on the schedule train is following). Once the train is put in the network, each train is either to move to the next resource
(station or track) or wait for some time at the current resource (predefined to 1 unit time, can be altered). These actions
help the train to complete it's journey from source to destination.
There can be multiple trains that need action at the same time.
TRAINS\_NEEDING\_ACTION is a global variable, that keeps track of the trains that need action at current simulation
time. So there are two tasks at hand :
\begin{itemize}
\item Choose a train from TRAINS\_NEEDING\_ACTION for taking an action. In this simulator, we are 
    using deadlock detection heuristic based on \cite{ARTICLE:2} for picking the train. Essentially this heuristic breaks
    the tie when multiple trains are waiting for taking the action.
    \vspace{-1cm}
    \item Next step is to take the action, either to move or to wait. Choice of action depends on the state
    space of the train (discussed in detail in Algorithm section). We can also randomize this process by 
    choosing the action randomly with fixed probabilities.
\end{itemize}
\item \textbf{Deadlock detection}

This process is invoked after every predefined time (20 units) and checks if the network is in deadlock 
or not.\textbf{ Banker's Algorithm} is used as the deadlock detection algorithm (discussed in detail in 
resource usage module). If the network is in deadlock, then 
simulation of the current episode terminates.

\item \textbf{Create Statistics}

This process is invoked after every predefined time (20 units) and generates statistics about the current 
state of the network in the main log file (look at log generation component). The statistics include :
\begin{itemize}
\item Number of trains not yet started. 
\item Number of trains currently running in the network.
\item Number of trains that have completed their journey but the resource is not freed.
\item Number of trains the have completed their journey and all the resources are freed.    
\end{itemize}
If all the trains have completed their journey and all resources are freed then the simulation is terminated.
More statistics about the state of the network can be added in future.  

\item \textbf{Update Graph}

This process is responsible for creating the running GIF of the railway network and the trains running on
the network. It's purpose is only visualization that further helps in debugging and analysis.
\end{enumerate}

All these processes are run by the simulator. In future, more processes can be added with different 
functionality. All one has to do is to create a component and then the simulator will create a process
that runs the component.


