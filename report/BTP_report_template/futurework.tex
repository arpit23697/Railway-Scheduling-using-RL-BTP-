\chapter{Conclusion and Future Work}

For implementing the RL algorithms to learn schedules, we need a discrete event simulator 
that drives the algorithm. So far, focus is to implement the simulator. Simulator is organised into different modules 
and these modules work together to create the whole system. Railway network and train module implements 
the underlying railway network and trains that run over these networks. Statistical analysis and graph module 
creates graphs for visualisation and generates log for analysis and debugging. More components 
will be added in future for more analysis as per the need. Resource usage module 
keeps track of the resources (station and track) in the network. It generates Resource usage graph, 
implements deadlock detection algorithm for detecting deadlock in the network and deadlock avoidance heuristic for choosing action when multiple
trains need action at the same time. Algorithm module forms the backbone of the whole simulator and contains 
implementation of RL algorithms (Q-learning and Deep Q-Learning) for learning the 
schedule. Finally simulator module, puts each piece together by creating each important 
component as process. Each train in the network is treated as process and runs simultaneously.
Other processes are also created for detecting deadlock, generating graph for visualisation,
checking status of the network (number of trains in the network and there status) and choosing 
action for trains that need action. 

\vspace{\baselineskip}
Now the future focus is on the algorithm module that will contain implementation of 
prior as well as proposed approach. The report discusses
two approaches, to solve the problem. Once the prior approach is implemented, we can
see how good it is working, how good it is scaling to real life problem instances, how good it is 
performing compared to the present approaches and how to improve upon it. Next step is to 
improve on this algorithm and add deep learning to it for learning the state space. Finally, 
we drive towards the proposed approach and improve upon it.  

\vspace{\baselineskip}
Since we are tackling blocking version of the JSSP problem, so the approach that we will 
develop can be used to solve the JSSP problem with reasonable approximation. So the future plan is also 
to use the developed approaches on other similar problems as well.
