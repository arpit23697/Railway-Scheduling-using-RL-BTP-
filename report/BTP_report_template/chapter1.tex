\titleformat{\chapter}[display]{\normalfont\Large\bfseries}{}{11pt}{\Huge}
\chapter{Introduction}
\pagenumbering{arabic}\hspace{3mm}


The aim is to work on an algorithm for scheduling
bidirectional railway network (both single- and multi-track) using a
framework of Reinforcement Learning. Given deterministic arrival/departure times for
 all the trains on the lines, their initial positions, 
 priority and halt times, traversal times, deciding on track allocations is a 
 job shop scheduling problem (NP Complete ). However, 
 due to the stochastic nature of the delays, 
 the track allocation decisions have to be made in a dynamic manner, 
 while minimising the total priority-weighted delay. 
 This makes the underlying problem one of decision making in of 
 stochastic event driven systems. 

 \vspace{\baselineskip}
First we will focus on solving scheduling problem for railway line and discuss different algorithm, there 
advantages, results. Then using the knowledge developed so far, we will solve the flatland environment\cite{ARTICLE:3} that is 
grid based simulator for multiagent reinforcement learning for any re-scheduling problem (RSP). 


\vspace{\baselineskip}
This report is organised in 7 main chapters. Chapter 2 presents the problem statement of railway scheduling on railway line, 
Chapter 3 provides details for the implementation of the simulator, Chapter 4 discusses the algorithm details for railway scheduling on railway line and 
Chapter 5 shows the experiments and results of the algorithms. Chapter 6 starts discussing the flatland problem, 
Chapter 7 discusses the 
Double deep q-learning based approach for flatland problem and presenting the results and Chapter 8 presents 
cooperative path finding approach for solving flatland environment and also presents the results. Last chapter presents the 
conclusion.
