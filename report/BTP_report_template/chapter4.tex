\chapter{Implementation Overview}

As for now, we are focussing on how to implement the first approach 
and then move onto the second approach. The integrated reinforcement learning algorithm is driven
by a discrete event simulator. There are already some railway simulators like \textbf{OpenTrack} \cite{WEBSITE:2} and \textbf{RailML}\cite{WEBSITE:3} but they
would be useful for the final analysis of the results. Once we have the desired timetable then we can use these 
simulator softwares to determine the quality of solution. But for the implementation of the 
algorithm we have to implement the simulator on own.

\section {Railway simulator}
\subsection{Requirement}
The simulator is supposed to be robust enough that it can run both toy and real life 
examples. 
The simulator is suppose to run through several episodes during training and hence need 
to be efficient. At the
beginning of every episode, the initial locations of all the trains
are reset to their original values. It is assumed that trains that have not yet started, 
or have finished their journeys, do not
occupy any of the tracks. Following the train-to-resource mapping, 
the simulator creates a list of events for processing, one
corresponding to each train (whether already running or yet to
start its journey). Each event in the list contains the following
information: the time at which to process the event, the train to
which it corresponds, the resource where the train is currently
located, the last observed state-action pair for the train (empty
if the train is yet to start), and the direction of the train journey. 

\vspace{\baselineskip}
At each step, the algorithm moves the simulation clock to
the earliest time stamp in the event list. If multiple events are to
be processed at the same time stamp, they are handled sequentially. We are for now 
not focussing on how to avoid deadlock, but instead if we get into deadlock, we will detect and give huge
negative reward and the RL algorithm is suppose to avoid deadlock on it's own.


\subsection{Implementation}
There are two components to railway simulator :
\begin{enumerate}
    \item Underlying Railway Network. 
    \item Trains and the simulation of there movements.
\end{enumerate}

For the implementation of the railway network we can use \textbf{NetworkX}\cite{WEBSITE:4} package of python.
\textbf{NetworkX is a Python package for the creation, manipulation,
and study of the structure, dynamics, and functions of complex networks.}
\vspace{1.5cm}

Once the network is ready we have to simulate movement of each train over the network. For 
that we can use \textbf{SimPy}\cite{WEBSITE:5} package of python. \textbf{SimPy is a process-based discrete-event simulation framework based on standard Python}.
In this we can model each train as the separate process and network as the resource. We can 
model the movement of trains using this package. We yield events when the train starts from some station and once 
the train reaches the next station, event is yielded and then we can process accordingly. So whole simulation is done 
by generating events at points where the algorithm is supposed to take action.

\vspace{\baselineskip}
Once the railway network is created, train class is used to create different instances
of the trains running over the network.